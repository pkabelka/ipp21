\documentclass[10pt,a4paper]{article}
\usepackage[utf8]{inputenc}
\usepackage[T1]{fontenc}
\usepackage[czech]{babel}
\usepackage[tmargin=1in,bmargin=1in,lmargin=1in,rmargin=1in]{geometry}
\usepackage{times}
\usepackage[none]{hyphenat}
% \usepackage[unicode, colorlinks, hypertexnames=false, citecolor=red]{hyperref}

\renewcommand{\baselinestretch}{1.2}

\begin{document}
    {\noindent\sffamily\large
        \textbf{Implementační dokumentace k 1. úloze do IPP 2020/2021} \\
        Jméno a~příjmení: Petr Kabelka \\
        Login: xkabel09
    }

    \section{Analyzátor}

    Skript \emph{parse.php} provádějící lexikální a~syntaktickou analýzu je
    rozdělen do 4~hlavních tříd, které provádějí činnosti od kontroly vstupního
    kódu IPPcode21 až po generování reprezentace kódu ve formátu XML.

    Zpracování začíná ve třídě \emph{Parser} a~její metodou \emph{parse}, ta
    nejprve zpracuje parametry se kterými byl skript spuštěn a~pokračuje
    vytvořením instancí tříd \emph{Output} a~\emph{Inst}. Oba tyto objekty
    jsou průběžně používány uvnitř cyklu \emph{while} který načítá instrukce
    tak dlouho, dokud nedojdou instrukce ke zpracování. Tento cyklus slouží
    k~zapsání jednotlivých instrukcí a jejich argumentů do XML reprezentace.
    Samotná analýza kódu probíhá v~metodě \emph{next} třídy \emph{Inst}.

    Načtení a~zpracování instrukce začíná načtením řádku ze standardního vstupu
    (STDIN) a~oříznutím komentářů a~nadbytečných bílých znaků. První kontrolou
    je nalezení hlavičky \emph{.IPPcode21} a~následně se řádek rozdělí podle
    mezer na část s~operačním kódem a~část s~argumenty. Kontrolu typů
    a~počtu argumentů se snadno provádí díky konstatní proměnné
    \emph{INST\_MAP}, která slouží jako šablona pro všechny zpracovávané
    instrukce. Každá instrukce je klíčem do tohoto asociativního pole a~každá
    má své pole s~typy operandů v~pořadí v~jakém je instrukce příjmá. Operační
    kód se kontroluje existencí klíče v~\emph{INST\_MAP}. Lexikální
    a~syntaktická správnost argumentů je ověřena kontrolou správného typu
    operandu a~jeho hodnot pomocí regulárních výrazů.

    Tvorbu výstupního XML zajišťuje třída \emph{Output}, která využívá
    knihovnu \emph{XMLWriter} pro sestavení elementů a~jejich atributů.

    \subsection{Rozšíření}

    V~rámci řešení 1.~projektu jsem implementoval rozšíření \textbf{STATP},
    umožňující sběr skupin statistik o~analyzovaném kódu. Třída \emph{Stats}
    obsahuje statické metody pro inkrementaci či dekrementaci hodnot statistik
    určených jejich názvem. Pro zápis do souborů statistik slouží metoda
    \emph{write}, která očekává asociativní pole kde klíče jsou názvy souborů
    a~hodnoty jsou pole jejichž položkami jsou názvy statistik.

    Kvůli kompatibilitě analyzátoru s~interpretem jsem také implementoval
    zásobníkové instrukce.

\end{document}
